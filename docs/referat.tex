\begin{center}
	\LARGE\bfseries{РЕФЕРАТ}
\end{center}

Расчетно-пояснительная записка \pageref{LastPage} с., \totalfigures{} рис., \totaltables{} таблицы, 16 источников, 0 приложений.

СТАТИЧЕСКИЙ СЕРВЕР, ВЕБ-СЕРВЕР, HTTP, NGINX, НАГРУЗОЧНОЕ ТЕСТИРОВАНИЕ, APACHE BENCHMARK.

Цель работы~--- реализация классического статического веб-сервера для отдачи контента с диска.

Была изучена предметная область, связанная со статическим веб-сервером.
Был проведен анализ протокола HTTP.
Была осуществлена формализация бизнес-правил разрабатываемого программного обеспечения.

Были сформулированы требования к разрабатываемому статическому веб-серверу.
Были проанализированы сокеты как средство взаимодействия между процессами.
Были разработаны схемы алгоритмов работы статического веб-сервера и обработки HTTP-запросов.

Были выбраны средства реализации статического веб-сервера.
Был написан код программного обеспечения для отдачи контента с диска по HTTP-запросу.

Было проведено сравнение результатов нагрузочного тестирования разработанного программного обеспечения и сервера, развернутого на базе nginx. Согласно полученным данным, сервер на базе nginx выполняет запросы в среднем в 2.5 раза быстрее, чем разработанное программное обеспечение.
