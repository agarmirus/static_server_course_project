\chapter{Технологическая часть}

Для реализации статического веб-сервера был использован язык программирования C.
Программное обеспечение было разработано для операционных систем на базе ядра Linux.
Для инициализации сервера были использованы системные вызовы socket, bind, listen и accept~\cite{socket_sys, bind, listen, accept}.
Для чтения запросов и записи ответов были применены функции recvfrom и sendto соответственно~\cite{recvfrom, sendto}.
Для реализации требуемой архитектуры были использованы системные вызовы fork и pselect~\cite{fork, pselect}.

В листинге~\ref{logger_code} представлена реализация логгера статического веб-сервера.
\begin{lstlisting}[label=logger_code,caption=Реализация логгера статического веб-сервера,language=Caml]
#include "logger.h"

int logger_init(
	logger_t *const logger_ptr,
	const char *log_file_name,
	const loglevel_t level
)
{
	logger_ptr->log_file_fd = open(log_file_name, O_WRONLY | O_CREAT, S_IRUSR | S_IWUSR | S_IRGRP | S_IROTH);
	
	if (logger_ptr->log_file_fd == -1)
		return EXIT_FAILURE;
	
	logger_ptr->level = level;
	
	return EXIT_SUCCESS;
}

void logger_close(logger_t *const logger_ptr)
{
	close(logger_ptr->log_file_fd);
}

static void print_msg(
	const int fd,
	const char *msg,
	const char *prefix
)
{
	char log_msg[MAX_LOG_MSG_LENGTH] = {0};
	
	time_t cur_local_time = time(NULL);
	
	char *cur_local_time_str = asctime(localtime(&cur_local_time));
	cur_local_time_str[strlen(cur_local_time_str) - 1] = 0;
	
	sprintf(
		log_msg,
		"[%s] %s: %s\n",
		cur_local_time_str,
		prefix,
		msg
	);
	
	write(fd, log_msg, strlen(log_msg) + 1);
}

void logger_fatal(
	logger_t *const logger_ptr,
	const char *msg
)
{
	if (logger_ptr->level >= FATAL)
		print_msg(logger_ptr->log_file_fd, msg, "FATAL");
}

void logger_error(
	logger_t *const logger_ptr,
	const char *msg
)
{
	if (logger_ptr->level >= ERROR)
		print_msg(logger_ptr->log_file_fd, msg, "ERROR");
}

void logger_warn(
	logger_t *const logger_ptr,
	const char *msg
)
{
	if (logger_ptr->level >= WARN)
		print_msg(logger_ptr->log_file_fd, msg, "WARN");
}

void logger_info(
	logger_t *const logger_ptr,
	const char *msg
)
{
	if (logger_ptr->level >= INFO)
		print_msg(logger_ptr->log_file_fd, msg, "INFO");
}

void logger_debug(
	logger_t *const logger_ptr,
	const char *msg
)
{
	if (logger_ptr->level >= DEBUG)
		print_msg(logger_ptr->log_file_fd, msg, "DEBUG");
}

void logger_trace(
	logger_t *const logger_ptr,
	const char *msg
)
{
	if (logger_ptr->level >= TRACE)
		print_msg(logger_ptr->log_file_fd, msg, "TRACE");
}
\end{lstlisting}

В листинге~\ref{httptools_code} представлена реализация обработчика HTTP-запросов.
\begin{lstlisting}[label=httptools_code,caption=Реализация обработчика HTTP-запросов,language=Caml]
#include "httptools.h"

#define REQ_LINE_PARTS_COUNT 2

#define READ_BUF_SIZE 200
#define VERB_BUF_SIZE 10
#define PATH_BUF_SIZE 100

#define DEFAULT_HEADERS_BUF_SIZE 10
#define HTTP_RES_STRING_BUF_SIZE 200

httpreq_t httpreq_parse_from_client(
	struct sockaddr_in *const client_address,
	socklen_t *const client_address_len,
	const int socket_fd
)
{
	httpreq_t httpreq = {UNKNOWN, NULL};
	
	char read_buf[READ_BUF_SIZE] = {0};
	
	if (recvfrom(socket_fd, read_buf, READ_BUF_SIZE, 0, (struct sockaddr *)client_address, client_address_len) == -1)
	return httpreq;
	
	char verb_buf[VERB_BUF_SIZE] = {0};
	char path_buf[PATH_BUF_SIZE] = {0};
	
	int read_count = sscanf(read_buf, "%s%s", verb_buf, path_buf);
	
	if (read_count < 2)
		return httpreq;
	
	if (strcmp(verb_buf, "GET") == 0)
		httpreq.verb = GET;
	else if (strcmp(verb_buf, "HEAD") == 0)
		httpreq.verb = HEAD;
	else
		return httpreq;
	
	if (strcmp(path_buf, "/") == 0)
		strcpy(path_buf, DEFAULT_FILE);
	
	size_t read_path_length = strlen(path_buf);
	
	char *new_path = calloc(read_path_length + 2, sizeof(char));
	
	if (new_path)
	{
		strcpy(new_path + 1, path_buf);
		*new_path = '.';
		
		httpreq.path = new_path;
	}
	
	return httpreq;
}

void httpreq_free(httpreq_t *const httpreq)
{
	free(httpreq->path);
	httpreq->path = NULL;
}

bool httpreq_is_null(const httpreq_t *const httpreq)
{
	return !(bool)httpreq->path;
}

httpverb_t httpreq_verb(const httpreq_t *const httpreq)
{
	return httpreq->verb;
}

const char *httpreq_path(const httpreq_t *const httpreq)
{
	return httpreq->path;
}

static httpheaders_t httpheaders_init(void)
{
	httpheaders_t httpheaders = {0, 0, NULL};
	
	return httpheaders;
}

httpres_t httpres_init(
	const short unsigned code,
	const char *description
)
{
	httpheaders_t httpheaders = httpheaders_init();
	httpres_t httpres = {0, NULL, httpheaders};
	
	size_t description_length = strlen(description);
	
	char *new_description = malloc(description_length);
	
	if (new_description)
	{
		memcpy(new_description, description, description_length);
		
		httpres.code = code;
		httpres.description = new_description;
	}
	
	return httpres;
}

static void httpheaders_free(httpheaders_t *const httpheaders)
{
	for (size_t i = 0; i < httpheaders->length; ++i)
	{
		free(httpheaders->arr[i].name);
		free(httpheaders->arr[i].value);
	}
	
	free(httpheaders->arr);
	httpheaders->arr = NULL;
}

void httpres_free(httpres_t *const httpres)
{
	free(httpres->description);
	httpres->description = NULL;
	
	httpheaders_free(&httpres->headers);
}

static int httpheaders_add_header(
	httpheaders_t *const httpheaders,
	const httpheader_t *const httpheader
)
{
	if (!httpheaders->arr)
	{
		httpheaders->arr = calloc(DEFAULT_HEADERS_BUF_SIZE, sizeof(httpheader_t));
		
		if (!httpheaders->arr)
		{
			return ERROR_HTTP_RES_ADD_HEADER;
		}
		
		httpheaders->buf_size = DEFAULT_HEADERS_BUF_SIZE;
	}
	
	if (httpheaders->length == httpheaders->buf_size)
	{
		httpheader_t *realloced_arr = realloc(httpheaders->arr, 2 * httpheaders->buf_size);
		
		if (!realloced_arr)
		{
			return ERROR_HTTP_RES_ADD_HEADER;
		}
		
		httpheaders->arr = realloced_arr;
		httpheaders->buf_size *= 2;
	}
	
	httpheaders->arr[httpheaders->length++] = *httpheader;
	
	return EXIT_SUCCESS;
}

int httpres_add_header(
	httpres_t *const httpres,
	const char *name,
	const char *value
)
{
	size_t name_length = strlen(name);
	size_t value_length = strlen(value);
	
	char *new_name = malloc(name_length);
	
	if (!new_name)
		return ERROR_HTTP_RES_ADD_HEADER;
	
	char *new_value = malloc(value_length);
	
	if (!new_value)
	{
		free(new_name);
		
		return ERROR_HTTP_RES_ADD_HEADER;
	}
	
	strcpy(new_name, name);
	strcpy(new_value, value);
	
	httpheader_t header = {new_name, new_value};
	
	int rc = httpheaders_add_header(&httpres->headers, &header);
	
	if (rc != EXIT_SUCCESS)
	{
		free(new_name);
		free(new_value);
	}
	
	return rc;
}

void httpres_send(
	const int socket_fd,
	struct sockaddr_in *const client_address,
	socklen_t *const client_address_len,
	const httpres_t *const httpres
)
{
	char buf[HTTP_RES_STRING_BUF_SIZE] = {0};
	
	sprintf(buf, "HTTP/1.1 %d %s\r\n", httpres->code, httpres->description);
	sendto(socket_fd, buf, strlen(buf), 0, (struct sockaddr *)client_address, *client_address_len);
	
	for (size_t i = 0; i < httpres->headers.length; ++i)
	{
		sprintf(buf, "%s: %s\r\n", httpres->headers.arr[i].name, httpres->headers.arr[i].value);
		sendto(socket_fd, buf, strlen(buf), 0, (struct sockaddr *)client_address, *client_address_len);
	}
	
	sendto(socket_fd, "\r\n", 2, 0, (struct sockaddr *)client_address, *client_address_len);
}
\end{lstlisting}

В листинге~\ref{pathtools_code} представлена реализация модуля для обработки строк, содержащий путь к файлу.
\begin{lstlisting}[label=pathtools_code,caption=Реализация модуля для обработки строк\, содержащий путь к файлу,language=Caml]
#include "pathtools.h"

filetype_t file_type(const char *path)
{
	char *type_str = strrchr(path, '.');
	
	if (!type_str)
		return NONE;
	
	++type_str;
	
	if (strcmp(type_str, FILE_TYPE_HTML) == 0)
		return HTML;
	
	if (strcmp(type_str, FILE_TYPE_CSS) == 0)
		return CSS;
	
	if (strcmp(type_str, FILE_TYPE_JS) == 0)
		return JS;
	
	if (strcmp(type_str, FILE_TYPE_PNG) == 0)
		return PNG;
	
	if (strcmp(type_str, FILE_TYPE_JPG) == 0)
		return JPG;
	
	if (strcmp(type_str, FILE_TYPE_JPEG) == 0)
		return JPEG;
	
	if (strcmp(type_str, FILE_TYPE_SWF) == 0)
		return SWF;
	
	if (strcmp(type_str, FILE_TYPE_GIF) == 0)
		return GIF;
	
	return NONE;
}

size_t file_size(const int fd)
{
	lseek(fd, 0, SEEK_SET);
	
	size_t size = (size_t)lseek(fd, 0, SEEK_END);
	
	lseek(fd, 0, SEEK_SET);
	
	return size;
}

bool path_is_inside(const char *path)
{
	char *path_copy = malloc((strlen(path) + 1) * sizeof(char));
	
	if (!path_copy)
	{
		return false;
	}
	
	memcpy(path_copy, path, (strlen(path) + 1) * sizeof(char));
	
	int dir_level = 0;
	
	char *dir_str = strtok(path_copy, "\\");
	
	while (dir_str)
	{
		if (strcmp(dir_str, ".") != 0)
			dir_level += strcmp(dir_str, "..") == 0 ? -1 : 1;
		
		dir_str = strtok(NULL, "\\");
	}
	
	return dir_level >= 0;
}
\end{lstlisting}

В листинге~\ref{fdsarr_code} представлена реализация массива файловых дескрипторов и операций над ним.
\begin{lstlisting}[label=fdsarr_code,caption=Реализация массива файловых дескрипторов и операций над ним,language=Caml]
#include "fdsarr.h"

fdsarr_t fdsarr_alloc(const size_t buf_size)
{
	fdsarr_t fdsarr = {0, 0, NULL};
	
	int *arr = calloc(buf_size, sizeof(int));
	
	if (arr)
	{
		fdsarr.buf_size = buf_size;
		fdsarr.arr = arr;
	}
	
	return fdsarr;
}

void fdsarr_free(fdsarr_t *const fdsarr)
{
	free(fdsarr->arr);
	
	fdsarr->arr = NULL;
	fdsarr->length = fdsarr->buf_size = 0;
}

bool fdsarr_is_null(const fdsarr_t *const fdsarr)
{
	return !(bool)fdsarr->arr;
}

int fdsarr_append(fdsarr_t *const fdsarr, const int elem)
{
	if (fdsarr->length == fdsarr->buf_size)
	{
		int *realloced_arr = realloc(fdsarr->arr, 2 * fdsarr->length);
		
		if (!realloced_arr)
			return EXIT_FAILURE;
		
		fdsarr->arr = realloced_arr;
		fdsarr->buf_size *= 2;
	}
	
	fdsarr->arr[fdsarr->length++] = elem;
	
	return EXIT_SUCCESS;
}

void fdsarr_remove(fdsarr_t *const fdsarr, const int elem)
{
	size_t i = 0;
	
	for (; i < fdsarr->length && fdsarr->arr[i] != elem; ++i);
	
	if (i < fdsarr->length)
	{
		for (size_t j = i; j < fdsarr->length - 1; ++j)
		fdsarr->arr[j] = fdsarr->arr[j + 1];
		
		--fdsarr->length;
	}
}

void fdsarr_close(const fdsarr_t *const fdsarr)
{
	for (size_t i = 0; i < fdsarr->length; ++i)
		close(fdsarr->arr[i]);
}

size_t fdsarr_length(const fdsarr_t *const fdsarr)
{
	return fdsarr->length;
}

int fdsarr_max(const fdsarr_t *const fdsarr)
{
	if (fdsarr->length == 0)
		return -1;
	
	int max_fd = fdsarr->arr[0];
	
	for (size_t i = 1; i < fdsarr->length; ++i)
		if (max_fd < fdsarr->arr[i])
			max_fd = fdsarr->arr[i];
	
	return max_fd;
}
\end{lstlisting}

В листинге~\ref{server_code} представлена реализация статического веб-сервера.
\begin{lstlisting}[label=server_code,caption=Реализация статического веб-сервера,language=Caml]
#include <fcntl.h>
#include <stdlib.h>
#include <signal.h>
#include <sys/stat.h>
#include <sys/socket.h>
#include <sys/select.h>
#include <netinet/in.h>

#include "conf.h"
#include "logger.h"
#include "fdsarr.h"
#include "serverror.h"
#include "httptools.h"
#include "pathtools.h"

#define READ_BUF_SIZE 100

#define max(a, b) a >= b ? a : b

static int listen_fd = 0;

logger_t logger = {0, 0};

void server_shutdown(int signum)
{
	logger_info(&logger, "server shutdown");
	
	logger_trace(&logger, "closing listen socket");
	close(listen_fd);
	
	logger_trace(&logger, "closing logger");
	logger_close(&logger);
	
	exit(EXIT_SUCCESS);
}

static void change_sig_handlers(void)
{
	logger_trace(&logger, "setting SIGINT signal handler");
	signal(SIGINT, server_shutdown);
	
	logger_trace(&logger, "setting SIGTERM signal handler");
	signal(SIGTERM, server_shutdown);
	
	logger_trace(&logger, "setting SIGTSTP signal ignoring");
	signal(SIGTSTP, SIG_IGN);
	
	logger_trace(&logger, "setting SIGCHLD signal ignoring");
	signal(SIGCHLD, SIG_IGN);
}

static void init_listen_fd(void)
{
	logger_trace(&logger, "initializing server socket address struct");
	
	struct sockaddr_in serv_addr;
	serv_addr.sin_family = AF_INET;
	serv_addr.sin_addr.s_addr = INADDR_ANY;
	serv_addr.sin_port = htons(SERVER_PORT);
	
	logger_trace(&logger, "creating listen cosket");
	listen_fd = socket(AF_INET, SOCK_STREAM, 0);
	
	if (listen_fd == -1)
	{
		logger_fatal(&logger, "error while creating listen socket");
		exit(ERROR_SOCKET_INIT);
	}
	
	logger_trace(&logger, "binding listen socket to server address");
	if (bind(listen_fd, (struct sockaddr *)&serv_addr, sizeof(serv_addr)) == -1)
	{
		logger_fatal(&logger, "error while creating listen socket");
		
		logger_trace(&logger, "closing listen socket");
		close(listen_fd);
		
		exit(ERROR_SOCKET_BIND);
	}
	
	logger_trace(&logger, "marking listen socket as connection-mode");
	if (listen(listen_fd, LISTEN_MAX_CONNECTIONS) == -1)
	{
		logger_fatal(&logger, "error while marking listen socket as connection-mode");
		
		logger_trace(&logger, "closing listen socket");
		close(listen_fd);
		
		exit(ERROR_LISTEN_SOCKET);
	}
}

static sigset_t init_pselect_sigset(void)
{
	logger_info(&logger, "initializing pselect sigset");
	
	sigset_t pselect_sigmask;
	
	sigemptyset(&pselect_sigmask);
	sigaddset(&pselect_sigmask, SIGTERM);
	
	return pselect_sigmask;
}

static void send_error(
	const int fd,
	struct sockaddr_in *const client_address,
	socklen_t *const client_address_len,
	const int code,
	const char *description
)
{
	logger_info(&logger, "sending HTTP-error");
	
	httpres_t res = httpres_init(code, description);
	
	char content[100] = {0};
	sprintf(content, "<h1>%d %s</h1>", code, description);
	
	char content_length_str[100] = {0};
	sprintf(content_length_str, "%zu", strlen(content));
	
	httpres_add_header(&res, "Content-Type", "text/html");
	httpres_add_header(&res, "Content-Length", content_length_str);
	
	httpres_send(fd, client_address, client_address_len, &res);
	
	sendto(fd, content, strlen(content), 0, (struct sockaddr *)client_address, *client_address_len);
}

static int form_httpres_headers(
	httpres_t *const httpres,
	const char *content_type,
	const int file_fd
)
{
	logger_info(&logger, "forming http headers");
	
	int rc = httpres_add_header(httpres, "Content-Type", content_type);
	
	if (rc != EXIT_SUCCESS)
	{
		return rc;
	}
	
	char buf[100] = {0};
	sprintf(buf, "%zu", file_size(file_fd));
	
	rc = httpres_add_header(httpres, "Content-Length", buf);
	
	return rc;
}

static void send_file_content(
	const int client_fd,
	struct sockaddr_in *const client_address,
	socklen_t *const client_address_len,
	const int file_fd
)
{
	logger_info(&logger, "sending file content");
	
	char buf[READ_BUF_SIZE] = {0};
	
	ssize_t read_bytes = read(file_fd, buf, READ_BUF_SIZE);
	
	while (read_bytes > 0)
	{
		sendto(client_fd, buf, read_bytes, 0, (struct sockaddr *)client_address, *client_address_len);
		
		read_bytes = read(file_fd, buf, READ_BUF_SIZE);
	}
}

static void send_response(
	const int client_fd,
	struct sockaddr_in *const client_address,
	socklen_t *const client_address_len,
	const int file_fd,
	const filetype_t filetype,
	const httpverb_t verb
)
{
	logger_info(&logger, "sending OK http response");
	
	logger_trace(&logger, "send_response - initializing httpes");
	httpres_t httpres = httpres_init(HTTP_CODE_OK, "OK");
	
	int rc = EXIT_SUCCESS;
	
	logger_trace(&logger, "send_response - generating httpres headers according to file type");
	switch (filetype)
	{
		case NONE:
			rc = ERROR_HTTP_UKNOWN_FILE_TYPE;
			break;
		case HTML:
			rc = form_httpres_headers(&httpres, HTTP_CONTENT_TYPE_HTML, file_fd);
			break;
		case CSS:
			rc = form_httpres_headers(&httpres, HTTP_CONTENT_TYPE_CSS, file_fd);
			break;
		case JS:
			rc = form_httpres_headers(&httpres, HTTP_CONTENT_TYPE_JS, file_fd);
			break;
		case PNG:
			rc = form_httpres_headers(&httpres, HTTP_CONTENT_TYPE_PNG, file_fd);
			break;
		case JPG:
			rc = form_httpres_headers(&httpres, HTTP_CONTENT_TYPE_JPG, file_fd);
			break;
		case JPEG:
			rc = form_httpres_headers(&httpres, HTTP_CONTENT_TYPE_JPEG, file_fd);
			break;
		case SWF:
			rc = form_httpres_headers(&httpres, HTTP_CONTENT_TYPE_SWF, file_fd);
			break;
		case GIF:
			rc = form_httpres_headers(&httpres, HTTP_CONTENT_TYPE_GIF, file_fd);
			break;
	}
	
	if (rc == EXIT_SUCCESS)
	{
		logger_trace(&logger, "send_response - generating httpres headers according to file type");
		httpres_send(client_fd, client_address, client_address_len, &httpres);
		
		if (verb == GET)
		{
			send_file_content(client_fd, client_address, client_address_len, file_fd);
		}
	}
	else
	{
		logger_info(&logger, "internal server error while generating httpres headers");
		send_error(client_fd, client_address, client_address_len, HTTP_CODE_INTERNAL_SERVER_ERROR, "Internal server error");
	}
	
	httpres_free(&httpres);
}

static void perform_request(
	const int client_fd,
	struct sockaddr_in *const client_address,
	socklen_t *const client_address_len,
	const httpreq_t *const httpreq_ptr
)
{
	logger_info(&logger, "performing http request");
	
	logger_trace(&logger, "perform_request - checking http verb");
	httpverb_t verb = httpreq_verb(httpreq_ptr);
	
	if (verb != GET && verb != HEAD)
	{
		logger_info(&logger, "sending 405 Method Not Allowed");
		send_error(client_fd, client_address, client_address_len, HTTP_CODE_METHOD_NOT_ALLOWED, "Method not allowed");
		return;
	}
	
	logger_trace(&logger, "perform_request - getting http URI");
	const char *path = httpreq_path(httpreq_ptr);
	
	if (!path_is_inside(path))
	{
		logger_info(&logger, "sending 403 Forbidden");
		send_error(client_fd, client_address, client_address_len, HTTP_CODE_FORBIDDEN, "Forbidden");
		return;
	}
	
	logger_trace(&logger, "perform_request - opening source file for reading");
	int requested_fd = open(path, O_RDONLY);
	
	if (requested_fd == -1)
	{
		logger_info(&logger, "sending 404 Not Found");
		send_error(client_fd, client_address, client_address_len, HTTP_CODE_NOT_FOUND, "Not found");
		return;
	}
	
	logger_trace(&logger, "perform_request - getting source file type");
	filetype_t filetype = file_type(path);
	
	if (filetype == NONE)
	{
		logger_info(&logger, "unknown file type, sending 405 Method Not Allowed");
		send_error(client_fd, client_address, client_address_len, HTTP_CODE_METHOD_NOT_ALLOWED, "Method not allowed");
		close(requested_fd);
		return;
	}
	
	send_response(client_fd, client_address, client_address_len, requested_fd, filetype, verb);
	
	logger_trace(&logger, "perform_request - closing source file");
	close(requested_fd);
}

static int serve_client(const int client_fd)
{
	logger_info(&logger, "forking proccess for client service");
	
	int pid = -1;
	
	if ((pid = fork()) == -1)
	{
		logger_error(&logger, "cannot fork");
		
		return ERROR_FORK_PROCCESS;
	}
	
	if (pid == 0)
	{
		logger_info(&logger, "child process is serving clients request");
		
		logger_trace(&logger, "serve_client - parsing http request from client");
		
		struct sockaddr_in client_address;
		socklen_t client_address_len = sizeof(client_address);
		
		httpreq_t httpreq = httpreq_parse_from_client(&client_address, &client_address_len, client_fd);
		
		if (httpreq_is_null(&httpreq))
		{
			logger_error(&logger, "cannot parse http request, child process is beeing killed");
			
			exit(ERROR_HTTP_REQ_PARSE);
		}
		
		logger_trace(&logger, "serve_client - getting http verb");
		httpverb_t verb = httpreq_verb(&httpreq);
		
		perform_request(client_fd, &client_address, &client_address_len, &httpreq);
		
		logger_trace(&logger, "serve_client - free http request structure memory");
		httpreq_free(&httpreq);
	}
	
	close(client_fd);
	
	if (pid == 0)
		exit(EXIT_SUCCESS);
	
	return EXIT_SUCCESS;
}

static int connect_new_client(
	fdsarr_t *const client_fds_arr_ptr
)
{
	logger_info(&logger, "accepting new connection");
	
	logger_trace(&logger, "connect_new_client - accepting new client fd with accept");
	struct sockaddr client_addr; 
	socklen_t client_len;
	
	int client_fd = accept(listen_fd, (struct sockaddr *)&client_addr, &client_len);
	
	if (client_fd == -1)
	{
		logger_error(&logger, "client socket accepting error");
		
		return ERROR_SOCKET_ACCEPT;
	}
	
	logger_trace(&logger, "connect_new_client - appending new client fd to clients fds array");
	int rc = fdsarr_append(client_fds_arr_ptr, client_fd);
	
	if (rc != EXIT_SUCCESS)
	{
		logger_error(&logger, "connect_new_client - cannot append new client fd to clients fds array");
		
		return ERROR_FDSARR_APPEND;
	}
	
	return EXIT_SUCCESS;
}

static void handle_read_sockets(
	fdsarr_t *const client_fds_arr_ptr,
	fd_set *const read_fd_set_ptr
)
{
	logger_info(&logger, "handling sockets");
	
	size_t i = 0;
	
	while (i < client_fds_arr_ptr->length)
	{
		int socket_fd = client_fds_arr_ptr->arr[i];
		
		if (FD_ISSET(socket_fd, read_fd_set_ptr))
		{
			FD_CLR(socket_fd, read_fd_set_ptr);
			fdsarr_remove(client_fds_arr_ptr, socket_fd);
			
			serve_client(socket_fd);
		}
		else
		++i;
	}
	
	if (FD_ISSET(listen_fd, read_fd_set_ptr))
		connect_new_client(client_fds_arr_ptr);
}

static void update_read_fds_set(
	int *const nfds_ptr,
	fd_set *const read_fd_set_ptr,
	const fdsarr_t *const client_fds_arr_ptr
)
{
	logger_info(&logger, "updating read fds set");
	
	FD_ZERO(read_fd_set_ptr);
	
	size_t client_fds_count = fdsarr_length(client_fds_arr_ptr);
	
	for (size_t i = 0; i < client_fds_count; ++i)
		FD_SET(client_fds_arr_ptr->arr[i], read_fd_set_ptr);
	
	FD_SET(listen_fd, read_fd_set_ptr);
	
	*nfds_ptr = max(listen_fd, fdsarr_max(client_fds_arr_ptr));
}

int main(void)
{
	int rc = logger_init(&logger, DEFAULT_LOG_FILE_NAME, DEFAULT_LOG_LEVEL);
	
	if (rc != EXIT_SUCCESS)
		return ERROR_LOGGER_INIT;
	
	logger_info(&logger, "setting signals handlers");
	change_sig_handlers();
	
	logger_info(&logger, "initializing listen socket");
	init_listen_fd();
	
	logger_info(&logger, "initializing read fd set");
	fd_set read_fd_set;
	FD_ZERO(&read_fd_set);
	FD_SET(listen_fd, &read_fd_set);
	
	logger_info(&logger, "allocation memory for clients fds array");
	fdsarr_t client_fds_arr = fdsarr_alloc(DEFAULT_FDSARR_BUF_SIZE);
	
	if (fdsarr_is_null(&client_fds_arr))
	{
		logger_fatal(&logger, "cannot allocate memory for clients fds array");
		
		logger_info(&logger, "closing listen socket");
		close(listen_fd);
		
		logger_info(&logger, "closing logger");
		logger_close(&logger);
		
		exit(ERROR_FDSARR_ALLOC);
	}
	
	sigset_t pselect_sigmask = init_pselect_sigset();
	
	int nfds = listen_fd;
	
	logger_info(&logger, "running server");
	
	while (1)
	{
		logger_info(&logger, "searching for fds with pselect");
		rc = pselect(nfds + 1, &read_fd_set, NULL, NULL, NULL, &pselect_sigmask);
		
		if (rc == -1)
		{
			logger_info(&logger, "closing listen socket");
			close(listen_fd);
			
			logger_info(&logger, "closing read fds from fdsarr");
			fdsarr_close(&client_fds_arr);
			
			logger_info(&logger, "free fdsarr allocated memory");
			fdsarr_free(&client_fds_arr);
			
			logger_info(&logger, "closing logger");
			logger_close(&logger);
			
			exit(ERROR_PSELECT_READ);
		}
		
		handle_read_sockets(&client_fds_arr, &read_fd_set);
		
		update_read_fds_set(&nfds, &read_fd_set, &client_fds_arr);
	}
	
	return EXIT_SUCCESS;
}
\end{lstlisting}

\section*{Вывод}

Были выбраны средства реализации статического веб-сервера.
Был написан код программного обеспечения для отдачи контента с диска по HTTP-запросу.

\clearpage