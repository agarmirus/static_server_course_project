\ssr{ЗАКЛЮЧЕНИЕ}

В результате выполнения работы был реализован классической статический веб-сервер для отдачи контента с диска.

Была изучена предметная область, связанная со статическим веб-сервером.
Был проведен анализ протокола HTTP.
Была осуществлена формализация бизнес-правил разрабатываемого программного обеспечения.

Были сформулированы требования к разрабатываемому статическому веб-серверу.
Были проанализированы сокеты как средство взаимодействия между процессами.
Были разработаны схемы алгоритмов работы статического веб-сервера и обработки HTTP-запросов.

Были выбраны средства реализации статического веб-сервера.
Был написан код программного обеспечения для отдачи контента с диска по HTTP-запросу.

Было проведено сравнение результатов нагрузочного тестирования разработанного программного обеспечения и сервера, развернутого на базе nginx. Согласно полученным данным, сервер на базе nginx выполняет запросы в среднем в 2.5 раза быстрее, чем разработанное программное обеспечение.

Были выполнены следующие задачи:
\begin{itemize}
	\item анализ предметной области, связанной со статическим веб-сервером;
	\item предъявление требований к разрабатываемому программному обеспечению;
	\item проектирование архитектуры статического веб-сервера для отдачи контента с диска;
	\item реализация статического веб-сервера для отдачи контента с диска;
	\item исследование характеристик реализованного сервера.
\end{itemize}
